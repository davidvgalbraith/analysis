\documentclass[11pt, oneside]{article}   	% use "amsart" instead of "article" for AMSLaTeX format
\include{davemathshortcuts}
\geometry{letterpaper}                   		% ... or a4paper or a5paper or ... 
%\geometry{landscape}                		% Activate for for rotated page geometry
%\usepackage[parfill]{parskip}    		% Activate to begin paragraphs with an empty line rather than an indent
\usepackage{graphicx}				% Use pdf, png, jpg, or eps§ with pdflatex; use eps in DVI mode
								% TeX will automatically convert eps --> pdf in pdflatex		
\usepackage{amssymb}

\title{Chapter 1: Real Numbers}
\author{Dave}
%\date{}							% Activate to display a given date or no date

\begin{document}
\maketitle

\be
\item If $x \in A \cup (B \cap C)$, then $x \in A$ or both $x \in B$ and $x \in C$. If $x \in A$, then $x \in (A \cup B) \cap (A \cup C)$. And if $x \not \in A$, then $x \in (A \cup B) \cap (A \cup C) \lra x \in B \cap C$. Thus any $x$ is either in both of the sets in question or neither, so they are the same set.
\item \be
\item $x \not \in A^c \ra x \in A$, so $(A^c)^c \subset A$, and $x \in A \ra x \not \in A^c$, so $A \subset (A^c)^c$. 
\item If $x$ is not in both $A$ and $B$, then either $x \not \in A$ or $x \not \in B$, so either $x \in A^c$ or $x\in B^c$, so $A^c \cup B^c \subset (A \cap B)^c$. And if $x$ is in $A^c \cup B^c$ then either $x \not \in A$ or $x \not \in B$, so $x$ is not in both $A$ and $B$, so $x \in (A \cap B)^c$. So $(A \cap B)^c = A^c \cup B^c$. Taking complements of both sides and applying part a, we get $A \cap B = (A^c \cup B^c)^c$. Applying part a again, we can consider $A^c$ to be our $A$ and $B^c$ to be our $B$, and it follows that $(A \cup B)^c = A^c \cap B^c$.
\item Yeah ok dude
\item The complement of the intersection of arbitrary sets is the union of the complements of those sets, and the complement of an arbitrary union of sets is the intersection of their complements. The proof is not substantively different from the two-set case.
\ee
\item \be
\item There does not exist a prime $p$ such that $p < 2$.
\item For any bounded plane region $A$ there is a line $y$ parallel to the $x$-axis such that $y$ bisects the area of $A$.
\item It is not the case that for all $x$, $x$ glitters $\ra x$ is gold.
\ee
\item In English, it either means that no matter how much hope the prisoner has, there's always room to hope more, or that no matter how little hope the prisoner has, that's too much hope and he should have less.
\item \be
\item It is not the case that if roses are red then violets are blue.
\item It is not the case that if he does not swim, he will sink.
\ee
\item An integer $n$ is even $\lra$ $n$'s prime factorization includes a 2 $\lra$ the prime factorization of any power of $n$ includes a 2 $\lra$ any power of $n$ is even.
\item \be
\item $n \in \ints$ is even $\ra n = 2k$ for some $k \in \ints \ra n^2 = 4k^2$.
\item The same $n$ from part a satisfies $n^3 = 8k^3$.
\item If $m=2k+1$ then $m^3 = 8k^3 + 12k^2 + 6k + 1$ which is odd, and two times an odd number is not divisible by 8.
\item Let $\frac{p}{q}$ be a fraction in lowest terms. Then $\frac{p^3}{q^3} = 2 \ra p^3 = 2q^3$, so $p$ must be even, so $q$ must be odd since $\frac{p}{q}$ was in lowest terms. So $p^3$ is divisible by 8, and $2q^3$ is twice an odd cube, so it follows from part c that there is no such $\frac{p}{q}$.
\ee
\item \be 
\item Suppose $k \in \nats$ is not a perfect $n$th power. Then $k^{1/n} \not \in \nats$, so if $k = p/q$ then $q$ does not divide $p$. But then $q^n$ does not divide $p^n$, so $k = (k^{1/n})^n = (\frac{p}{q})^n  \not \in \nats$. So $k$ cannot be $p/q$.
\item Inferred.
\ee
\item \be
\item Let $a \in A, b \in B, a' \in A', b' \in B'$, then $a < b$ and $a' < b'$ so $a + a' < b + b'$ so $A + A' $ is disjoint from $B + B'$. Let $A$ be the set of all $k \in \rats$ such that $k < 0$ or $k^2 < 2$, let $B$ be the rest of $\rats$, and let $A'$ be the the set of al $k \in \rats$ such that $k < 0$ and $(1-k)^2 < 2$, $B'$ be the rest of $\rats$. Then $A + A'$ is the set of rational numbers less than 1, but $B + B'$ is the set of rational numbers greater than 1, so neither of them contains 1.  
\item It does indeed follow that that definition would not represent a cut in all cases.
\item $A$ and $A'$ contain negative numbers of arbitrarily large magnitude, so $A * A'$ is all of $\rats$ in all cases which doesn't make for very interesting multiplication.
\ee
\item \be
\item Let $a$ be the least upper bound of $A$, then $C$ is the set of rationals less than $1/a$ and $D$ is the rest of $\rats$.
\item Let $a$ be the least upper bound of $A$, then $C$ is the set of rationals less than $-1/a$ and $D$ is the rest of $\rats$.
\item Suppose $xy = 1$ and $xy' = 1$, then $y' = (xy)y' = (xy')y = y$.
\ee
\item Given a real number $x$ and a real number $y > x$, $\frac{x+y}{2}$ is a real number between $x$ and $y$, so there is no smallest real number $> x$. Taking $x=0$, it follows that there is no smallest positive real number, and the same argument holds for $x$ and $y$ rational.
\item \be
\item If no such $s$ existed for a given $\epsilon$ then $b - \epsilon$ would be a lower upper bound of $S$, so there's no such $\epsilon$.
\item Nope, $\{1\}$ is a bounded nonempty subset of $\reals$ with no such $s$.
\item This exercise is complete gibberish. 
\ee
\item $x*x \le 2$ by the definition of $x$. So suppose $x^2 < 2$. Then there exists some $\epsilon$ such that $x^2= 2 - \epsilon$. For this $\epsilon$, there exists a $\delta > 0$ such that $\delta^2 + 2\delta < \epsilon$. So $(x + \delta)^2 = x^2 + 2\delta + \delta^2 < 2$. So $x+\delta$ is a larger upper bound for $A$ than $x$, a contradiction which shows that $x^2 = 2$.
\item If $n=1$ take $\delta = \epsilon$ and the result is immediate. So suppose for some arbitrary $n-1$ that $|u-y| < \delta \ra |u^{n-1} - y^{n-1}| < \epsilon$ for some $\delta$ given any $\epsilon$. Then $|u^n - y^n| = |u-y||\sum_{i=0}^{n-1}u^iy^{n-1-i}| \le |u-y|\sum_{i=0}^{n-1}|u^iy^{n-1-i}|$. Substituting $u = y - \delta$ or $u + \delta$ into this latter equation, we get that $|u^n - y^n| \le \delta(ny^{n-1} +$ [some polynomial in $\delta$]). We can choose $\delta$ such that this expression becomes arbitrarily small.
\item Let $y$ be the least upper bound of $S = \{s \in \reals: s^n \le x\}$, then if $y^n = x + \epsilon$ then by exercise 14 there exists a $\delta > 0$ such that $|y - u| < \delta \ra |y^n - u^n| < \epsilon$ for all $u \in \reals$, so $x < (y - \delta)^n < y^n$, but $y - \delta \in S$ so this can't be the case. Likewise, if $y^n = x -\epsilon$, then there exists a $\delta$ such that $x - \epsilon < (y - \delta)^n < x$, so $y$ is not an upper bound of $S$ as we supposed it to be. By trichotomy, $y^n = x$.
\item \be
\item $0 \le x - N < 1 \ra 0 \le 10(x - N) < 10) \ra 0 \le x_1 \le 9$. So suppose the first $n-1$ of the $x_i$ all obey $0 \le x_i \le 9$, then $x_n = 10^nx - 10^nN - \sum_{i=1}^{n-1}10^{n-i}x_i$. If $x_n < 0$, then $10^nx - 10^nN - \sum_{i=1}^{n-1}10^{n-i}x_i < 0 \ra 10^{n}(x - N - \sum_{i=1}^{n-1}\frac{x_i}{10^i}) < 0 \ra 10^{n-1}(x - N - \sum_{i=1}^{n-1}\frac{x_i}{10^{i-1}}) < 0 \ra 10^{n-1}(x-N-\sum_{i=1}^{n-2}\frac{x_i}{10^{i-1}}) < x_{n-1}$ which is in contradiction to the definition of $x_{n-1}$. Similarly, if $x_n \ge 10$ then $10^nx - 10^nN - \sum_{i=1}^{n-1}10^{n-i}x_i \ge 10 \ra 10^{n-1}(x-N-\sum_{i=1}^{n-2}\frac{x_i}{10^{i-1}}) \ge x_{n-1} + 1$, again in contradiction to the definition of $x_{n-1}$. So it must be the case that $0 \le x_n \le 9$. 
\item Were there some $k$ such that there did not exist $l > k$ with $x_l \not = 9$, $x$ would terminate in an infinite string of nines, which is outside the realm of possibility by hypothesis.
\item Consider the set in question as a sequence $\{a_n\}$. Then $a_k \le x$ for all $k$, and if $i$ is the first index such that $(x-\epsilon)_i < x_i$, then $a_i > x - \epsilon$. So $x$ is the least upper bound of this sequence.
\item Repeat the above argument with the arbitrary base $b$ in place of 10.
\ee
\item The greatest lower bound of a set $X$ is the $g \in \reals$ such that $\forall x \in X, x \ge g$, and if $g'$ is some other lower bound of $X$, then $g' \le g$. $g$ is the greatest lower bound of $X \lra -g$ is the least upper bound of $\{-x: x \in X\}$, so the greatest lower bound property is equivalent to the least upper bound property.
\item \be
\item No pair of pairs $\{(a, b), (a', b')\}$ has $a=a'$. This is the "vertical line test" for functioniness.
\item The graph of $f$ is drawn in the $A-B$ plane, while the graph of $g$ is drawn in the $B-C$ plane. The graph of $g \circ f$ is in the $A-C$ plane. To try to relate the graphs somehow, you can project the graph of $f$ along the $B$-axis and apply $g$ to the result to get an $(a, c) \in g \circ f(A)$.
\ee
\item \be
\item Tautology.
\item Let $X = \{x \in [0, 1]: f(x) \ge x$ and there does not exist $x' < x: f(x') < x'\}$. $X$ is nonempty because it contains 0. $X$ is bounded above by 1, so it has a least upper bound $c$. I claim $f(c) = c$. Suppose $f(c) < c$. Then $f(c) = c - \epsilon$ for some $\epsilon$. By continuity, there exists a $\delta$ with $\epsilon > \delta > 0$ such that $|x-c| < \delta \ra |f(x) - f(c)| < \epsilon - \delta$. (That's because given a $\delta > 0$ for some $\epsilon$, any $\delta' < \delta$ also satisfies the continuity condition, so we can take $\delta' > 0$ such that $\epsilon - \delta'$ is arbitrarily close to $\epsilon$). Therefore, for $x \in [c - \delta, c], f(x) \le c - \epsilon + \epsilon - \delta = c - \delta$, so $c- \delta$ is a lower upper bound for $X$ than $c$. On the other hand, if $f(c) = c + \epsilon$, then for the same $\delta$ as above, $f(c + \delta) \ge c + \epsilon - \epsilon + \delta = c + \delta$, so $c$ is not an upper bound of $X$. 
\item Nope, take $f(x) = x^2$ and you'll see that it has fixed points at 0 and 1 but none in $(0, 1)$. 
\item Hell no, that's obscene.
\ee
\item A cube contains the ball with the same center and diameter the length of the cube's sides; this is clearly the largest ball that the cube contains. More generally, a box contains the ball whose diameter is the box's shortest side's length. The largest cube inscribed in a ball is one whose diagonal is a diameter of the ball.
\item \be
\item The only way for arbitrary squares $A$ and $B$ to intersect that is not listed among the potential ways for dyadic squares to intersect is for the squares to intersect in two points. The intersection characteristics of a pair of squares are invariant under affine transformations and rotations of the pair, so we can without loss of generality treat the case where $A$ is the unit square with bottom left corner at the origin, and $B$ intersects its left edge, the segment $[0, 0] \ra [0, 1]$. Then the left edge of $B$ must have $x$-coordinate less than 0, and the right edge of $B$ must have $x$-coordinate greater than 0. But $p/2^k < 0 \lra (p+1)/2^k \le 0$, so no dyadic square can meet this requirement, so $A$ and $B$ can only intersect in the listed ways.
\item The same proof holds without adaptation in arbitrary dimension, except that the boxes intersect in $m$ points instead of strictly 2.
\ee
\item \be
\item We will prove that for $r < 1$, the disk of radius $r$ can be completely covered by a finite number of dyadic squares with intersection along their boundaries. The claim will follow from this because given $\epsilon > 0$ we can find $r$ such that $\pi - \pi r^2 < \epsilon$, completely cover the disk of radius $r$, and have less than $\epsilon$ left over. We're gonna need a herd of llamas to help us out.

Llama A: Any interval contains a dyadic subinterval.
\begin{proof}
Let $[a, b]$ be an interval in $\reals$. Let $k$ be an integer such that $\frac{1}{2^{k-1}} < b-a$, and let $m = \lceil 2^ka \rceil$. Inescapably, $a \le \frac{m}{2^k} < \frac{m+1}{2^k} \le b$, since $2^kb$ is at least $2^ka + 2$ which is at least $\lceil 2^ka \rceil + 1$. So $[\frac{m}{2^k}, \frac{m+1}{2^k}]$ is a dyadic subinterval of $[a, b]$.
\end{proof}

Llama B: Given an interval $[a, b]$ and $\epsilon > 0$, there exists a finite set of dyadic subintervals of $[a, b]$ disjoint except at their boundaries such that the total length of the dyadic intervals exceeds $a - b - \epsilon$.
\begin{proof}
Using Llama A, we can find an interval $[\frac{m}{2^k}, \frac{m+1}{2^k}] \subset [a, b]$. To the left of this interval, we can add intervals $[\frac{m-1}{2^k}, \frac{m}{2^k}]$ as long as $\frac{m-1}{2^k} > a$ (don't include the first one where $\frac{m-1}{2^k} < a$ in our set). Similarly, we can append intervals $[\frac{m+1}{2^k}, \frac{m+2}{2^k}]$ as long as $\frac{m+2}{2^k} < b$. The total length of the intervals in this set is at least $b - a - 2/2^k$, since otherwise we could have added another interval at one of the ends. So we can split up each of these intervals down the middle into two intervals of length $1/2^{k+1}$, add intervals onto the ends as before, and end up with a total length of at least $b - a - 1/2^{k + 1}$. Continuing like that until $1/2^{k+n} < \epsilon$, we achieve the result.
\end{proof}

Llama C: Any square of area $A$ contains finitely many dyadic squares such that the total area of the dyadic squares exceeds $A - \epsilon$ for any $\epsilon$. 
\begin{proof}
Let $d$ be the side length of our square $S$. Then use Llama B to cover one side of the square with $N$ dyadic intervals of total length at least $d - \frac{\epsilon}{2d}$. Let each of these intervals be one side of a dyadic square contained in $S$. Pile up a total of $\frac{d-\frac{\epsilon}{2d}}{2^k}$ dyadic squares of the same size on top of these squares. This is as many as you can fit in $S$. The area not covered by these squares is at most $d * \frac{\epsilon}{2d} * 2 = \epsilon$.
\end{proof}

So given $r < 1$, let $d = \frac{2 - 2r}{\sqrt{2}}$. By geometry, a square $S$ of side length $d$ whose center is on the disk of radius $r$ does not exceed the confines of the unit disk. Furthermore, a square of side length $d$ centered on our circle encloses at least $d$ arc length of the circle, because the arc of the circle enters and exits opposite sides. So we can cover as much of the circle as can be covered by squares of side length $d$ in at most $2\pi r / d$ contiguous squares. And the remainder can be entirely enclosed in a single square of side length less than $d$. And so we can cover each of these squares to arbitrary precision with dyadic squares using Llama C. Given this finite set of dyadic squares completely covering the arc of our circle, all we need do to cover the interior with dyadic squares is extend the lines of all our dyadic squares. This yields a grid of rectangles all of whose sides are some multiple of $1/2^k$ apart for some $k$, and we can easily subdivide a finite set of rectangles like that into a finite set of dyadic squares. So we've completely covered the disk of radius $r$, and as discussed way back there it follows that we can cover the disk of radius 1 to arbitrary precision.
\item Llama A remains true if you replace $[a, b]$ with $(a, b)$. It follows that Llama B remains true if you demand the intervals be disjoint. It follows that Llama C remains true if you demand the squares be disjoint. The theorem then follows from these modified lemmas just as it did before.
\item The unit sphere contains finitely many disjoint dyadic cubes of total volume $\frac{4\pi}{3}$, and the unit $n$-hypersphere contains finitely many $n$-hypercubes of total volume (volume of the unit $n$-hypersphere) - $\epsilon$, for any $\epsilon > 0$. 
\item The argument given in part a readily generalizes to an arbitrary region of $\reals^2$. That's because even though we used the circular properties to calculate the exact number of squares needed, the existence of the finite covering by squares only ultimately depended on the curve having finite length. So given the unit square, the inscribed unit circle covers $\pi/4$ of its area. Given $\epsilon > 0$, we can cover the remaining $1-\pi/4$ of the square by a finite set of disjoint dyadic squares of total volume at least $1 - \pi / 4 - \epsilon$. Inscribing a unit circle in each of these squares, we fill $\pi/4(1 - \pi / 4 - \epsilon \pi / 4)$. So the remaining left over uncovered area is $(\pi/4)^2 + \epsilon\pi/4$. Repeating the process, we can achieve an uncovered area of $(\pi/4)^k + \epsilon$ for any $k$ for any $\epsilon$, which becomes infinitesimal. 
\ee
\item \be
\item $s(R)$ approaches the circumference of the circle of radius $R$ $(2\pi R)$, while $b(R)$ approaches its area, $\pi R^2$, so the ratio tends to $2/R \ra 0$. $s(R)^2$ on the other hand goes to $4 \pi^2 R^2$, so the ratio tends to 4$\pi$. 
\item The interesting exponent is $\frac{m+1}{m}$, since that's when the exponents of $R$ will cancel out and leave constant factors of interest. Other exponents make the limit of the ratio infinite or 0.
\item $c(R) \sim b(R) - s(R)$, so $\frac{c(R)}{b(R)} \to \frac{b(R) - s(R)}{b(R)} \to 1$. 
\item Makes no difference in the limit.
\ee
\item \be
\item $m=1$ is an interval, which has two points and one segment. $m=2$ is a square, which has four points, four segments and one square. $m=3$ is a cube, which has eight points, twelve segments, six squares, and one cube.
\item When you transform an $m$-cube of side length $d$ into an $m+1$-cube, the process is that you draw another $m$-cube $d$ units away in the new direction, and you draw segments connecting the corresponding vertices of your new and old cube. This gives you twice as many points (0-cubes) and additional segments equal to the number of previous segments (for the new $m$-cube) plus the number of previous vertices (for the connecting lines). And the new number of faces is double the old number of faces plus the previous number of lines, for the same reason. In general, $f(k, m)$ = $f(k-1, m-1) + 2f(k, m-1)$, and $f(k, k) = 1$ for all $k$.
\item The $m=0$ column corresponds to a single point, so it has one point and no higher-dimensional cubes.
\item We'll proceed by induction on $m$. If $m=0$, the proof is trivial (1=1). So we make the induction hypothesis that for some $m$, $\sum_{k=0}^m(-1)^kf(k, m) = 1$. Then $\sum_{k=0}^{m+1}(-1)^kf(k, m+1) = \sum_{k=0}^{n+1}(-1)^k[f(k-1, m) + 2f(k, m)] = \sum_{k=0}^{m+1}(-1)^kf(k-1, m) + 2\sum_{k=0}^{m+1}(-1)^kf(k, m)$. Since $f(-1, m) = f(m+1, m) = 0$, this comes out to $\sum_{k=0}^m(-1)^{k+1}f(k, m) + \sum_{k = 0}^m(-1)^kf(k, m) = -1 + 2 = 1$ by the induction hypothesis. I love it when a proof comes together.
\ee
\item $s, t \in [0, 1], s + t = 1 \ra a \le sa + tb \le b$, so the segment is a subset of the interval. And given $x \in [a, b]$, let $s = \frac{b-x}{b-a}, t = 1-s$, then we have $0 \le s, t \le 1$, $s + t = 1$, and $sa + tb = a\frac{b-x}{b-a} + b(1-\frac{b-x}{b-a}) = \frac{(a-b)(b-x)}{b-a} + b = x - b + b = x$. So the interval is a subset of the segment, so it is the segment.
\item \be
\item For $m=1$, the proof is trivial. For $m=2$, we get back the definition of convexity. So suppose the claim holds up to some $m$. Then if $x$ is a linear combination of points in some convex set $X$, $x = \sum_{i=0}^{m+1}s_ix_i = s_0x_0 + \sum_{i=1}^{m+1} s_ix_i \ra x - s_0x_0 = \sum_{i=1}^{m+1}s_ix_i \ra \frac{x - s_0x_0}{\sum_{i=1}^{m+1}s_i} = \frac{\sum_{i=1}^{m+1}s_ix_i}{\sum_{i=1}^{m+1}s_i}$. So the right hand side is a convex combination of only $m$ points, so it is just some $x' \in X$. Then $x = \sum_{i=1}^{m+1}s_i(s_0x_0 + x')$, a convex combination of two elements of $X$, so $x \in X$. 
\item Because if it holds for any $m$ then it holds for $m=2$ so the set is convex.
\ee
\item \be
\item Let $\angles{(x, y, z), (x', y', z')} = \frac{1}{a^2}xx' + \frac{1}{b^2}yy' + \frac{1}{c^2}zz'$. Then $\angles{}$ is positive definite since $\angles{(x, y, z), (x, y, z)} = \frac{1}{a^2}x^2 + \frac{1}{b^2}y^2 + \frac{1}{c^2}z^2$ is a linear combination of nonnegative elements which is zero if and only if $(x, y, z) = (0, 0, 0)$. It's symmetric, because $\angles{(x, y, z), (x', y', z')} = \frac{1}{a^2}xx' + \frac{1}{b^2}yy' + \frac{1}{c^2}zz' + \frac{1}{a^2}x'x + \frac{1}{b^2}y'y + \frac{1}{c^2}z'z = \angles{(x', y', z'), (x, y, z)}$. And it's linear because $\angles{(x, y, z), (x' + x'', y'+ y'', z' + z'')} = \frac{1}{a^2}x(x'+x'') + \frac{1}{b^2}y(y' + y'') + \frac{1}{c^2}z(z'+z'') = \frac{1}{a^2}xx' + \frac{1}{b^2}yy' + \frac{1}{c^2}zz' + \frac{1}{a^2}xx'' + \frac{1}{b^2}yy'' + \frac{1}{c^2}zz'' = \angles{(x, y, z), (x', y', z')} + \angles{(x, y, z), (x'', y'', z'')}$. So $\angles{}$ is an inner product. The set in question is the unit ball for this inner product, so it is convex by the Cauchy Schwartz inequality, which applies to all inner products.
\item Intervals are convex by Exercise 25. So let $\{x_i\}$ be a set of intervals whose cross product forms a box, and consider elements $A, B = (a_1, a_2, \ldots a_n), (b_1,b_2, \ldots b_n)$ where each $a_i, b_i \in x_i$. Then each element of any convex combination of $A$ and $B$ is equal to $sa_i + tb_i \in x_i$. It follow that the convex combination of $A$ and $B$ is in the box defined by the $x_i$, so the box is convex. 
\ee
\item \be
\item If $f$ is not convex, then there exist $x, y, s, t \in \reals, 0 \le s, t \le 1, s + t = 1$ such that $f(sx + ty) > sf(x) + tf(y)$. The segment from $(x, f(x))$ to $(y, f(y))$ has its endpoints in $S$ but contains $(sx+ty, f(sx+ty))$ which is not in $S$, so $S$ is not convex. If $f$ is convex, then there don't exist such $x, y, s, t$, so $S$ contains every segment between points on the graph of $f$. And a line between arbitrary points in $S$ is the line between some $(a, f(a) + c_1)$ and $(b, f(b) + c_2)$ for some $c_1, c_2 \ge 0$, and $s(f(a) + c_1) + t(f(b) + c_2) = sf(a) + tf(b) + sc_1 + tc_2 \ge sf(a) + tf(b)$, so the segment between $(a, f(a) + c_1)$ and $(b, f(b) + c_2)$ is in $S$, so $S$ is convex.
\item If $f$ is convex, then $f(x + 1/n) \le \frac{n-1}{n}f(x) + \frac{1}{n}f(x+1)$. So given $\epsilon > 0$, we can take $n$ such that $1/n < \frac{\epsilon}{f(x+1) - f(x)}$ and get $f(x + 1/n) - f(x) \le \frac{1}{n}(f(x+1) - f(x)) \le \epsilon$. A similar argument applied to $f(x - 1/n)$ gives continuity of $f$.
\item Let $x < x' < u, x' = sx + tu$ for some $0 < s, t < 1$. If the slope $m$ of the line from $f(x)$ to $f(u)$ exceeds the slope $m'$ of the line from $f(x')$ to $f(u)$, then $f(x') > sf(x) + tf(u)$, because $f(u) - (sf(x) + tf(u)) = m(u-x') > m'(u - x') = f(u) - f(x')$. So $f$ cannot be convex if $m > m'$. A similar argument leads to the same conclusion for $x < u < u'$.
\item If $f''(x) < 0$, then by high school calculus there exists some interval $[x,x+\epsilon]$ such that $f(x + \epsilon / 2) - f(x) > f(x + \epsilon) - f(x + \epsilon / 2)$. By part c, this is disallowed for convex functions $f$.
\item Apply the definition we proved valid in part a to the graph of $f$. 
\ee
\item \be
\item Llama A: If $X$ is an uncountable set of positive numbers, then $\sum_{x \in X} x = \infty$.
\begin{proof}
Suppose $\sum_{x \in X} x$ is finite. If infinitely many $x \in X$ exceed $1/n$ for any $n \in \nats$, then $\sum_{x \in X} x \ge \infty * 1/n = \infty$. So all but finitely many $x \in X$ are less than $1/n$ for any $n \in \nats$. So $X = \cup_{n \in \nats}\{x \in X: x > 1/n\}$ is a countable union of finite sets, so it's countable. Therefore, if $X$ is uncountable, its sum is not finite.
\end{proof}

Let $C$ be described by $f(x): x \in \reals$ for some continuous $f$. The points where $C$ has no tangent line are the points where $f$ is not differentiable. Since $f$ is continuous, the points where it is not differentiable are the points where its left derivative does not equal its right derivative. At these points, the graph of $C$ forms a sharp corner with angle $0^\circ < \theta < 180^\circ$. Since $C$ is convex, all of these angles are in the same direction with respect to the positive $x$-axis, either clockwise or counterclockwise. And their total is $\le 360$, because $C$ is closed and convex. By Llama A, then, there are only countably many such corners, since each one corresponds to a positive rotation and their sum is finite.
\item The graph of a convex function is a subset of a closed curve like in part a, one with only 180 available degrees of rotation as we formulated them in part a, so it too can have only countable points of undifferentiability.
\ee
\item \be
\item A left limit at a point $c \in \reals$ is a number $L$ such that $\fred c - x < \delta \ra L-f(x) < \epsilon$. I assert that a monotone function $f$ has a left limit at all points $c \in [a, b]$. That's because let $L$ be the least upper bound of $\{f(x) : x < c\}$. Then $L$ fails to be a lower limit only if $\exists \epsilon > 0 : \forall \delta > 0 \exists x < c: L - f(x) > \epsilon$. Since $f$ is monotone, this implies that $L - \epsilon > f(x)$ for all $x < c$, so $L - \epsilon$ is a lower upper bound of $\{f(x): x < c\}$. Since we supposed that $L$ was the least upper bound, it follows that a monotone $f$ has a left limit at all $c \in [a, b]$, and a symmetric argument applied to greatest lower bounds shows that $f$ has a right limit as well. Therefore, the only discontinuities of $f$ are points where the left and right limits are not equal. In this case, since $f$ is monotone, the right limit must exceed the left limit by some $j > 0$. Since $f(b) < \infty$, this set must be countable by Llama A from exercise 29. 
\item The total number of discontinuities of $f$ is $\sum_{n \in \ints}$(number of discontinuities of $f$ on $[n, n + 1))$. By part a, the number of discontinuities of $f$ on each of these intervals must be countable. So the set of $f$'s discontinuities is a countable union of countable sets, so it's countable.
\ee
\item $f(i, j+1) = \frac{i^2 + (j + 1)^2 + i(2(j+1) - 3) - (j + 1) + 2}{2} = \frac{i^2 + j^2 + 2j + 1 + 2ij + 2i}{2} = \frac{i^2 + j^2 + 2ij - 3i - j + 2}{2} + i + j = f(i, j) + i + j$, and $f(i + 1, j) = \frac{(i + 1)^2 + j^2 + (i + 1)(2j - 3) - j + 2}{2} = \frac{i^2 + 2i + 1 + j^2 + 2ij - 3i + 2j - 3 - j + 2}{2} = \frac{i^2 + j^2 + 2ij - 3i - j + 2}{2} + i + j - 1 = f(i, j) + i + j  -1$. These are the same rules as followed by the table on page 33 enumerating the bijection from $\nats^2 \to \nats$, so $f$ is indeed an explicit form of this bijection.
\item A denumerable set is countable, so Corollary 18 immediately gives this result. The union of denumerably many countable sets could be finite if all but finitely many of them were empty and the nonempty ones were all finite.
\item \be
\item $\sim$ is an equivalence relation and $[a, b] \sim \reals \sim (a, b] \sim \reals \sim (a, b)$.
\item The same proof as a applies to the general case.
\item $\rats$ is a countable subset of the uncountable $\reals$, so $\reals \setminus \rats$ is uncountable.
\ee
\item The digit-shuffling map described in the hint gives an injection from $\reals^2$ to $\reals$, since if neither of $(x, y)$ terminates in an infinite string of nines than the shuffled digits of $x$ and $y$ won't terminate in an infinite string of nines. And $a \to (a, 0)$ is an injection from $\reals$ to $\reals^2$, so it follows from the Schroeder Bernstein Theorem that $\reals \sim \reals^2$.
\item \be
\item $\{s \in S: f(s) = 1\}$ is a subset of $S$ uniquely identified by $f$, and for every subset of $S$ there exists a function $f$ from $S$ to $\{0, 1\}$ such that $f(x) = 1 \lra x \in S$, so this is a natural bijection.
\item Suppose there exists a bijection between $S$ and $P$. Then there is a bijection $\beta: S \to F$, and for each $s\in S$, $\beta(s)$ is a function $f_s: S \to \{0, 1\}$. So for $t \in S,$ let $\gamma(t) = 0$ if $f_t(t) = 1$ and 0 if $f_t(t) = 1$. Then $\gamma$ disagrees with $\beta(t)$ for all $t$, and yet $\gamma$ is a perfectly valid function from $S$ to $\{0, 1\}$, so $\beta$ cannot have been a bijection.
\ee
\item \be
\item $\sum_{i=0}^na_ix_i \to (a_0, a_1, \ldots a_n)$ is a bijection from the polynomials with integer coefficients of degree $\le n$ to $\ints^{n+1}$, so denumerably many polynomials with integer coefficients have degree $n$ or less for all $n$ since $\ints^n$ is countable. So the set of all polynomials with integer coefficients is the union over all $n \in \nats$ of the polynomials of degree $\le n$, so there are denumerably many polynomials with integer coefficients since $\nats$ is countable. Since each polynomial of degree $n$ has at most $n$ distinct roots, the set of algebraic numbers is denumerable.
\item The only property of $\ints$ that mattered was its denumerability, so replace $\ints$ in the proof above with an arbitrary countable set $S$ and the proof still holds. 
\item By the fundamental theorem of trigonometric algebra as stated on page 77 of John P. Boyd's Solving Transcendental Equations, a trigonometric polynomial of degree $N$ has $2N$ roots on an interval $[a, a + \pi)$. It follows from the same argument we used in part a that the set of roots of trigonometric polynomials is countable.
\ee
\item \be
\item There are finitely many words of each length $n$ and countably many lengths $n$, so the set of strings is countable.
\item Same as a.
\item Same as a, except now there are countably many words of each length $n$ and countably many lengths $n$.
\item Associate each letter with a natural number mod $n$ and you have a bijection from this set to the set of decimal expansions base $n$, which has the cardinality of $\reals$.
\item The first, third and fourth $\sim$s follow from part d. For the second, use the map $(0, 0) \to 0, (0, 1) \to 1, (1, 0) \to 2, (1, 1) \to 3$ to convert two decimal expansions base $2$ to one decimal expansion base 4 and back.
\item Countably many decimal expansions terminate in an infinite string of nines, because you can associate each one that does with the finite word preceding the infinite string of nines, and by part a the set of finite words is countable. It follows from exercise 33 that uncountably many expansions do not terminate in an infinite string of nines.
\ee
\item Let $X = \{x \in [a, b]: f(x) \le v\}$. By the least upper bound property, $X$ has a least upper bound $L$, since it is bounded above by $b$. We have $L \le b$ because $f$ is not defined outside $[a, b]$ so $b$ is a lower upper bound for $X$ than any $L' > b$. And we have $L \ge a$ since we assumed that $v$ is a value taken by $f$ somewhere in $[a, b]$. And we have $f(L) = v$ since if $f(L) = v + \epsilon$ then by continuity of $f$ there exists a $\delta > 0$ such that for $x \in [L - \delta, L], f(x) \ge v$, so $L- \delta$ is a lower value than $L$ that nonetheless bounds every element of $X$ that $L$ bounds, and if $f(L) = v -\epsilon$ then there exists a $\delta > 0$ such that $f(L + \delta) \le \epsilon$, so $L$ is not an upper bound of $X$. It follows from these observations that $L$ Is the least upper bound of $X$, and a symmetric argument with the greatest lower bound property shows the other half of the claim.
\item \be
\item It's completely trivial that uniform continuity implies continuity. Continuity does not imply uniform continuity because $f(x) = 1/x$ is continuous on $(0, 1)$ but given $\epsilon = 1$ and any $\delta > 0$ you can take $0 < x < 1$ such that $\frac{x}{1-x} - x = \frac{x^2}{1-x} < \delta$ and $t = \frac{x}{1-x}$ then $|1/t - 1/x| = |\frac{1-x}{x} - \frac{1}{x}| = |\frac{-x}{x}| = 1$. 
\item Yes, given $\epsilon > 0$ take $\delta = \epsilon / 2$, then $|x - t| < \delta \ra f(x) - f(t) = 2(x - t) < 2\delta = \epsilon$.
\item No, take $\epsilon = 1$ then given $\delta > 0$ we have $f(x + \delta) - f(x) = x^2 + 2x\delta + \delta^2 - x^2 = x\delta + \delta^2$ so we can take $x > \frac{1-\delta^2}{2\delta}$ and then $f(x + \delta) - f(x) > \epsilon$.
\ee
\item Given $\epsilon > 0$, let $A(\delta) = \{u \in [a, b]: x, t \in [a, u], |x - t| < \delta \ra |f(x) - f(t)| < \epsilon\}$. Let $A = \cup_{\delta>0}A(\delta)$, and suppose $b \not \in A$. Then for all $\delta > 0$, $|f(b) - f(b - \delta)| > \epsilon$. But that contradicts the continuity of $f$, so $b \in A$, a, so $f$ is uniformly continuous on $[a, b]$.
\item Following the construction as in the proof of the Schroeder Bernstein, we get the following definition for our bijection $h$: if $k$ is the largest integer such that $n$ is a multiple of $2^k$ (i.e. $k=0$ if $n$ is odd, $k = 1$ if $n = 2, 6, 10, \ldots$, $k = 2$ if $n = 4, 12, 20, \ldots$), then $h(n) = 2n$ if $k$ is even and $n/2$ if $k$ is odd. $h$ turns out to be its own inverse, so it must be a bijection.
\item \be
\item Let $A_k = \{a_i: i \ge k\}$. If $(a_n)$ is bounded, then each $A_k$ is bounded, so each $A_k$ has a least upper bound $l_k$. So let $X = \{x \in \reals: x$ is the least upper bound of some $A_k \}$. Then $X$ is a nonempty, bounded set, so it has a greatest lower bound $l$. So $l_k \ge l$ for all $k$, and any other $l' \in \reals$ satisfying $l_k \ge l' \forall k$ also satisfies $l' < l$, so $l$ is the largest real number such that the supremum of infinitely many $A_k$ sets is at least $l$, so $l = \lim_{n \to \infty}(\sup_{k \ge n}a_k)$.
\item It must be $\infty$ since for any finite $m$, for any $k \in \nats$ there exists a $k' > k \in \nats$ such that $a_{k'} > m$. (for if this were not the case, max$\{a_1, a_2, \ldots a_k\}$ would be an upper bound for $(a_n)$). So the definition of the lim sup implies it's infinite in this case.
\item It must be $-\infty$ since for any finite $m$, for any $k \in \nats$ there exists a $k' > k \in \nats$ such that $a_{k'} < m$. (for if this were not the case, min$\{a_1, a_2, \ldots a_k\}$ would be a lower bound for $(a_n)$). So the definition of the lim sup implies it's $-\infty$ in this case.
\item The first one holds whenever it is not the case that one of the lim sups is $\infty$ and the other is $-\infty$, since $a_n \le l_1, b_n \le l_2 \forall n \ra a_n + b_n \le l_1 + l_2$. The second one holds in all cases.
\item lim inf $(a_n)$ = $\lim_{n \to \infty}\inf_{k \ge n}a_k$.
\ee
\item Skipped in the name of progress.
\item \be
\item $|x + y|^2 + |x-y|^2 = \angles{x+y, x+y} + \angles{x-y, x-y} = \angles{x, x} + \angles{x, y} + \angles{y, x} + \angles{y, y} + \angles{x, x} - \angles{x, y} - \angles{y, x} + \angles{y, y} = 2\angles{x, x} + 2\angles{y, y} = 2|x|^2 + 2|y|^2$.
\item Let $\angles{x, y} = |\frac{x + y}{2}|^2 - |\frac{x-y}{2}|^2$. It is manifest that this is positive definite and symmetric, and $\angles{x, x} = |\frac{x+x}{2}|^2 + |\frac{x - x}{2}|^2 = |x|$. 
\be
\item $\angles{x + y, z} + \angles{x - y, z} = |\frac{(x + y) + z}{2}|^2 - |\frac{(x + y) - z}{2}|^2 + |\frac{(x - y) + z}{2}|^2 - |\frac{(x - y) - z}{2}|^2 = (|\frac{(x + y) + z}{2}|^2 + |\frac{(x - y) + z}{2}|^2) - (|\frac{(x + y) - z}{2}|^2 +  |\frac{(x - y) - z}{2}|^2)$. Let $u = \frac{x + z}{2}, u' = \frac{x-z}{2}, v = \frac{y}{2}$, then this comes out to $(|u + v|^2 + |u - v|^2) - (|u' + v|^2 + |u' - v|^2)$, which is equal to $2(|u|^2 + |v|^2 - |u'|^2 - |v|^2)$ by the parallelogram law. So this simplifies to $2(|u|^2 - |u'|^2) = 2|\frac{x + z}{2}|^2 - 2|\frac{x - z}{2}|^2 = 2\angles{x, z}$. It follows that $\angles{}$ is bilinear in the first variable, so by symmetry it is bilinear in the second.
\item First we'll show inductively that $\angles{mx, y} = m\angles{x, y}$ for $m \in \nats$. So for $m=1$ we get $\angles{1x, y} = 1\angles{x, y}$ which is trivial. So suppose $\angles{mx, y} = m\angles{x, y}$, then $\angles{(m+1)x, y} = \angles{mx, y} + \angles{x, y}$ by the bilinearity from part i, which by the induction hypothesis is $(m+1)\angles{x, y}$. Furthermore, by bilinearity $\angles{-x, y} + \angles{x, y} = \angles{0, y} = 0$, so $\angles{-x, y} = -\angles{x, y}$. Furthermore $n * \angles{\frac{x}{n}, y} = \angles{x, y}$, since $\sum_{i=1}^n\angles{\frac{x}{n}, y} = \angles{\frac{nx}{n}, y} = \angles{x, y}$, so for $p \in \rats, p\angles{x, y} = \angles{px, y}$. Finally, for fixed $x, y$, let $f(\lambda) = \lambda\angles{x, y} - \angles{\lambda x, y}$, then $f(\lambda) = 0$ for $\lambda \in \rats$, and $f$ is continuous because both of its summands are continuous as a function of $\lambda$, so $f$ is uniformly 0.
\ee
\ee
\ee

\end{document}